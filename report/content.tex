\section{Introduction}
In the context of the personal project course given at ULiège. I decided to develop a smart freezer. This idea emerges due do a personal experience. Every year my family and I reorganize the home 's freezers. I was a bit shocked when I see the quantity of products which were in the freezers and of which I was not aware. The problem is even a product is in a freezer it cannot stay there indefinitely due to product properties. As results, some products were thrown to the rubbish. it could have been avoided if we had been aware of the existence of that products. The solution of that problem would be an application or a tool that gives the possibility to manage the content of your freezers and more.

\subsection{Explanation of the project itself}
The idea itself is a freezer manager application. The application will give you information about the products stored, the period since they are there. The products stored will linked only to a unique user due to the fact that most of the products stored in a freezer are \textit{home-made products}.\\
The application gives the possibility to directly find what you want with a specific nomenclature. Each product has a serial number by box and by freezer (if the user has several freezers). Each time a product comes in/out of the freezer the user needs to notify this manipulation to the application.\\
Moreover, the application has the possibility to do suggestions to the user according to his previous consumption and depending on what his / her freezer (s) contain for some time. The goal being that the application helps the user in his choices and also remind him what he has in his freezer and what should perhaps be consumed in first. \\

For this project the application is restricted to a web application but can be easily ported to others terminals due to the fact that the back-end part is based on a REST api which is cross-platform.

Front-end part and back-end part a aborder
\section{Content storing}
parler des hash de passwords des token etc.
\subsection{Discussion}
\subsection{Relational database}

Explication du projet


